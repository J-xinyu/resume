% !TEX TS-program = xelatex
% !TEX encoding = UTF-8 Unicode
% !Mode:: "TeX:UTF-8"

\documentclass{resume}
\usepackage{zh_CN-Adobefonts_external} % Simplified Chinese Support using external fonts (./fonts/zh_CN-Adobe/)
%\usepackage{zh_CN-Adobefonts_internal} % Simplified Chinese Support using system fonts
\usepackage{linespacing_fix} % disable extra space before next section
\usepackage{cite}

\begin{document}
\pagenumbering{gobble} % suppress displaying page number

\name{吉新宇}

\basicInfo{
  \email{jixinyu666@gmail.com} \textperiodcentered\ 
  \phone{(+86) 155-9281-6678} \textperiodcentered\ 
}
 
\section{\faGraduationCap\  教育背景}
\datedsubsection{\textbf{清华大学}}{2023 -- 至今}
\textit{联合培养硕士研究生}\ 智能网联车, 预计 2026 年 3 月毕业
\datedsubsection{\textbf{陕西理工大学}}{2023 -- 至今}
\textit{硕士研究生}\ 智能网联车, 预计 2026 年 3 月毕业
\datedsubsection{\textbf{陕西理工大学}}{2019 -- 2023}
\textit{学士}\ 车辆工程

\section{\faUsers\ 实习/项目经历}
\datedsubsection{\textbf{清华大学}}{2022年5月 -- 至今}
\role{实习}{导师: 张新钰\ 赵建辉\ 赵虚左}
机器人与无人驾驶实习生
\begin{itemize}
  \item 使用ROS2从0搭建一辆自主导航小车
  \item 使用Moveit2驱动机械臂
  \item 部署Autoware.Universe自动驾驶框架到实车上
  \item 使用ROS2结合Tensorrt和CUDA以及ONNX去部署yolov5,yolox,unet,ufldv2等算法
  \item 调研学习bevfusion算法及其部署,写教程文档
\end{itemize}

\datedsubsection{\textbf{航天三院}}{2021年5月 -- 2021年11月}
\role{实习}{导师: 苏沛东\ 余涛}
校园特使
\begin{itemize}
  \item 在校内做ROS, OpenCV, 深度学习, 机器学习, QT, SLAM等相关培训
  \item 辅助开发赛事流程及内容
\end{itemize}

\datedsubsection{\textbf{阿克曼无人车系统设计}}{2021年1月 -- 2021年8月}
\role{C/C++, Linux, Python}{个人项目}
\begin{onehalfspacing}
基于ROS1和阿克曼底盘搭建一辆无人驾驶车
\begin{itemize}
  \item 使用传统方法和目标检测方法实现红绿灯识别
  \item 使用Unet实现车道线分割
  \item 使用Socket实现图像的传输
  \item 使用相机模型实现单目测距
  \item 使用Tensorrt和DeepStream实现模型的部署
  \item 使用Navigation框架实现小车的自动导航及避障
\end{itemize}
\end{onehalfspacing}

\datedsubsection{\textbf{多车协同}}{2022 年1月 -- 2022年4月}
\role{Gazebo, URDF, ROS}{个人项目}
\begin{onehalfspacing}
基于ROS和Gazebo仿真的多车协同控制
\begin{itemize}
  \item 使用Gazebo仿真环境搭建多个移动机器人的描述模型
  \item 使用Leader-Follower算法实现多个移动机器人的编队控制
  \item 支持 FontAwesome 4.5.0
\end{itemize}
\end{onehalfspacing}

\datedsubsection{\textbf{自主泊车}}{2022 年4月 -- 2022年6月}
\role{C++, XML, LGSVL}{个人项目}
\begin{onehalfspacing}
LGSVL模拟器中实现基于Autoware.auto的AVP自主泊车算法
\begin{itemize}
  \item 使用ROS2搭建的无人驾驶软件
  \item 使用LGSVL搭建的环境仿真和车辆仿真
  \item 使用Autoware.auto提供的AVP算法实现自主泊车
\end{itemize}
\end{onehalfspacing}
% Reference Test
%\datedsubsection{\textbf{Paper Title\cite{zaharia2012resilient}}}{May. 2015}
%An xxx optimized for xxx\cite{verma2015large}
%\begin{itemize}
%  \item main contribution
%\end{itemize}

\section{\faCogs\ IT 技能}
% increase linespacing [parsep=0.5ex]
\begin{itemize}[parsep=0.5ex]
  \item 编程语言: C++, Python, CUDA, XML, YAML
  \item 平台: Ubuntu
  \item 开发: ROS1/ROS2, Tensorrt, Autoware, Git
\end{itemize}

\section{\faHeartO\ 获奖情况}
\datedline{\textit{金奖},第三届中国智能网联汽车国际大会CIVC}{2023年7月}
\datedline{\textit{国家二等奖},全国大学生智能车竞赛航天物流组}{2023年6月}
\datedline{\textit{国家二等奖},全国大学生计算机设计大赛无人驾驶组}{2021年11月}
\datedline{\textit{国家三等奖},全国大学生智能车竞赛航天物流组}{2021年6月}
\datedline{\textit{省级二等奖},全国大学生智能车竞赛智慧餐厅组}{2021年6月}
\datedline{\textit{省级二等奖},全国大学生智能车竞赛航天物流组}{2022年6月}


\section{\faInfo\ 自我评价}
% increase linespacing [parsep=0.5ex]
\begin{itemize}[parsep=0.5ex]
  \item 从2019年起就开始学习无人驾驶相关的技术栈,冥冥之中感觉自己就是干这个的,但是自学路上走了很多歪路,好在让自己学会了该如何正确的去学习一门技术。但是对于产品量产落地相关的技术找不到学习渠道,因此产生强烈的欲望希望接触产品,参与制造年轻人的第一辆汽车。
\end{itemize}

%% Reference
%\newpage
%\bibliographystyle{IEEETran}
%\bibliography{mycite}
\end{document}
